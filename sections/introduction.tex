%! TEX root = ../note-clt-sample-quantile.tex

\section{Introduction}

This note provides an exposition of the central limit theorem for sample
quantiles in the spirit of \citet[Example 24,
p. 53]{1984pollardConvergenceStochasticProcesses}.
A more general result can be found in \citet[Section
2.3.3, pp. 77-84]{1980serflingApproximationTheoremsMathematical}.
The purpose of this note is to provide a practical example of the
Lindeberg-Feller central limit theorem (CLT).

\begin{definition}
A function \(F : \mathbb{R} \to \mathbb{R}\) is a cumulative distribution
function (cdf) if and only if
\begin{enumerate}
  \item \(F\) is weakly increasing, i.e. if \(x \leq y\) then \(F (x) \leq F
    (y)\),
  \item \(F\) is right continuous, i.e. \(\lim_{x \downarrow y} F (x) = F (y)\)
    for every \(y \in \mathbb{R}\),
  \item \(\lim_{y \to - \infty} F (y) = 0\) and \(\lim_{y \to \infty} F (y) =
    1\).
\end{enumerate}
\end{definition}

\begin{definition}
For real valued random variables \(Y_{1}, \dots, Y_{n}\), the empirical
cumulative distribution function (ecdf) is defined by
\begin{equation*}
  \widehat{F}_{n} (y) := \widehat{F}_{n} \left( y; Y_{1}, \dots, Y_{n} \right)
  := \frac{1}{n} \sum_{i = 1}^{n} \mathbf{1} \left\{ Y_{i} \leq y \right\}.
\end{equation*}
\end{definition}

\begin{definition}
For a cdf \(G\) on \(\mathbb{R}\) and \(p \in (0, 1)\), the associated quantile
is
\begin{equation*}
  Q (p; G) := \inf \{y \in \mathbb{R} : G (y) \geq p\}.
\end{equation*}
\end{definition}

\begin{assumption}
\label{asm--iid}
Let \(F\) be a cumulative distribution function (cdf) on \(\mathbb{R}\).
\(\left\{ Y_{i} \right\}_{i = 1}^{\infty}\) is a sequence of iid random
variables all with cdf \(F\).
\end{assumption}

%%% Local Variables:
%%% mode: LaTeX
%%% TeX-master: "../note-clt-sample-quantile"
%%% End:

% LocalWords:  cdf ecdf
