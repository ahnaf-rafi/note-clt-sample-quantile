%! TEX root = ../note-clt-sample-quantile.tex

\section{Introduction}

This note revisits a classic asymptotic normality result for sample quantiles
which has been known since \citet{1946mostellerSomeUsefulInefficient} and
\citet{1949smirnovLimitDistributionsTerms}.
The approach given here follows those of
\citet[Chapter III Section 4 Example 24,
p. 53]{1984pollardConvergenceStochasticProcesses} and
\citet[Section 2.3.3, pp. 77-84]{1980serflingApproximationTheoremsMathematical}.
In particular, we relate the sampling distribution of the sample
quantile to probabilities determined by binomial sums whose probability
parameters may drift with the number of trials.
This drift phenomenon motivates the use of the classic central limit
theorem (CLT) for independent triangular arrays due to Liapunov.

Let \(\left\{ Y_{i} \right\}_{i = 1}^{\infty}\) be a sequence of independent and
identically distributed (iid) real valued random variables on a common
probability space \((\Omega, \mathscr{F}, \Pr)\).
Denote their common cumulative distribution function (cdf) [also called the
population cdf] by
\begin{equation*}
  F (y) := \Pr \left\{ Y_{1} \leq y \right\}.
\end{equation*}
The parameter of interest are the associated population quantiles, which are
\begin{equation}
  \theta_{p} := Q (p; F) := \inf \{y \in \mathbb{R} : F (y) \geq p\},
  \quad \text{for } p \in (0, 1).
  \label{eqn--quantile-fn}
\end{equation}
The empirical cdf (ecdf) corresponding to the first \(n\) observations is
\begin{equation*}
  \widehat{F}_{n} (y) := \widehat{F}_{n} \left( y; Y_{1}, \dots, Y_{n} \right)
  := \frac{1}{n} \sum_{i = 1}^{n} \mathbf{1} \left\{ Y_{i} \leq y \right\},
  % \label{eqn--ecdf-def}
\end{equation*}
with corresponding sample quantiles
\begin{equation}
  \widehat{\theta}_{p, n} := \inf \left\{ y \in \mathbb{R} : \widehat{F}_{n}
  (y) \geq p \right\} = Q \left( p; \widehat{F}_{n} \right) \quad \text{for } Q
  \text{ defined in \eqref{eqn--quantile-fn}.}
  \label{eqn--sample-quantile-def}
\end{equation}
We are interested in limits of the sampling distribution
\begin{equation}
  H_{p, n} (t) := \Pr \left\{ \sqrt{n} \left( \widehat{\theta}_{p, n} -
  \theta_{p} \right) \leq t \right\} = \Pr \left\{ \widehat{\theta}_{p, n} \leq
  \theta_{p} + \frac{t}{\sqrt{n}} \right\}.
  \label{eqn--sample-quantile-law-def}
\end{equation}
To characterize limits of \(H_{p, n} (\cdot)\) in
\eqref{eqn--sample-quantile-law-def}, we will sometimes need directional
differentiability assumptions on \(F\).
These are defined in \Cref{def--dir-diff} below.
\Cref{thm--clt-quantile} provides a statement of the main
asymptotic normality result.
\Cref{cor--clt-quantile} states the more commonly known form of the result for
absolutely continuous \(F\) with density \(f\) continuous at \(\theta_{p}\).

\begin{definition}
\label{def--dir-diff}
The function \(F : \mathbb{R} \to \mathbb{R}\) is right-differentiable at
\(y\) with right-derivative \(f (y)\) iff for every real sequence \(\left\{
\delta_{n} \right\}\) such that \(\delta_{n} > 0\) for all \(n \in \mathbb{N}\)
and \(\lim_{n \to \infty} \delta_{n} = 0\),
\begin{equation}
  \lim_{n \to \infty} \frac{F \left( y + \delta_{n} \right) - F (y)}{\delta_{n}}
  = f (y).
  \label{eqn--dir-diff-def}
\end{equation}
The function \(F\) is left-differentiable at \(y\) with left-derivative \(f
(y)\) if \eqref{eqn--dir-diff-def} holds instead for every
real sequence \(\left\{ \delta_{n} \right\}\) such that \(\delta_{n} < 0\) for
all \(n \in \mathbb{N}\) and and \(\lim_{n \to \infty} \delta_{n} = 0\).
\end{definition}

\begin{remark}
\label{rem--diff-equiv-bi-dir-diff}
\(F\) is differentiable at \(y\) with derivative \(f (y)\) iff \(F\) is both
right- and left-differentiable as in \Cref{def--dir-diff} and its left- and
right-derivatives are both equal to \(f (y)\).
\end{remark}

\begin{theorem}
\label{thm--clt-quantile}
Suppose the population cdf \(F\), and \(p \in (0, 1)\) and \(\theta_{p} \in
\mathbb{R}\) satisfy \(p = F \left( \theta_{p} \right)\).
Then
\begin{equation}
  \lim_{n \to \infty} H_{p, n} (0) = \frac{1}{2}.
  \label{eqn--clt-quantile-0}
\end{equation}
Suppose in addition to the hypothesis \(p = F \left( \theta_{p} \right)\),
one of the following holds:
\begin{enumerate}[label=\roman*.]
  \item \(F\) is right-differentiable at \(\theta_{p}\) with right-derivative
    \(f \left( \theta_{p} \right)\), and \(t > 0\);
  \item \(F\) is left-differentiable at \(\theta_{p}\) with left-derivative \(f
    \left( \theta_{p} \right)\), and \(t < 0\).
\end{enumerate}
Then
\begin{equation}
  \lim_{n \to \infty} H_{p, n} (t) = \Phi \left( \frac{f \left( \theta_{p}
  \right) t}{\sqrt{p (1 - p)}} \right).
  \label{eqn--clt-quantile-non0}
\end{equation}
Therefore if in addition to the hypothesis \(p = F \left( \theta_{p} \right)\),
\(F\) is differentiable at \(\theta_{p}\) with derivative \(f \left( \theta_{p}
\right)\), then
\begin{equation}
  \sqrt{n} \left( \widehat{\theta}_{p, n} - \theta_{p} \right) \rightsquigarrow
  \N \left( 0, \sigma^{2} (p) \right) \quad \text{as } n \to \infty, \text{
  where } \sigma^{2} (p) := \frac{p (1 - p)}{f \left( \theta_{p} \right)^{2}}.
  \label{eqn--clt-quantile}
\end{equation}
\end{theorem}

\begin{corollary}
\label{cor--clt-quantile}
If \(F\) is absolutely continuous with Lebesgue-density \(f\) and \(f\) is
continuous at \(\theta_{p}\), then \eqref{eqn--clt-quantile} also holds,
i.e.
\begin{equation*}
  \sqrt{n} \left( \widehat{\theta}_{p, n} - \theta_{p} \right) \rightsquigarrow
  \N \left( 0, \sigma^{2} (p) \right) \quad \text{as } n \to \infty, \text{
  where } \sigma^{2} (p) := \frac{p (1 - p)}{f \left( \theta_{p} \right)^{2}}.
\end{equation*}
\end{corollary}

The rest of this note is organized as follows.
\Cref{cor--clt-quantile} is proven in \Cref{sec--prf--cor--clt-quantile}.
\Cref{thm--clt-quantile} is proven in
\Cref{sec--prf--thm--clt-quantile}.
\Cref{sec--cdf-behavior} states and proves some fundamental properties of cdfs
that are used throughout these sections.
\Cref{sec--prf--thm--bin-clt-normal} proves a key central limit theorem
(\Cref{thm--bin-clt-normal}) used in the proof of \Cref{thm--clt-quantile}.

%%% Local Variables:
%%% mode: LaTeX
%%% TeX-master: "../note-clt-sample-quantile"
%%% End:

% LocalWords:  cdf ecdf
