%! TEX root = ../note-clt-sample-quantile.tex

\section{Some basic results on the behavior of cdfs}
\label{sec--cdf-behavior}

\begin{lemma}
\label{lem--p-leq-Fy-iff-Qp-leq-y}
Given \((p, y) \in (0, 1) \times \mathbb{R}\),
\begin{equation}
  p \leq F (y) \iff Q (p; F) \leq y.
  \label{eqn--p-leq-Fy-iff-Qp-leq-y}
\end{equation}
\end{lemma}

\begin{proof}[Proof of \Cref{lem--p-leq-Fy-iff-Qp-leq-y}]
\(p \leq F (y) \implies Q (p; F) \leq y\) is immediate from the definition of
\(Q\).
By \Cref{lem--FQ-geq-p} \(p \leq F (Q (p; F))\) and so if \(Q (p; F) \leq y\)
then \(p \leq F (Q (p; F)) \leq F (y)\).
\end{proof}

\begin{lemma}
\label{lem--FQ-geq-p}
For every \(p \in (0, 1)\), \(F (Q (p; F)) \geq p\).
\end{lemma}

\begin{proof}[Proof of \Cref{lem--FQ-geq-p}]
Suppose that for some \(p \in (0, 1)\), \(F (Q (p; F)) < p\).
Take any sequence \(\left\{ y_{n} \right\}\) such that \(Q (p; F) < y_{n + 1}
\leq y_{n}\) for all \(n \in \mathbb{N}\) and \(\lim_{n \to
\infty} y_{n} = Q (p; F)\).
Since \(F\) is increasing, \(F (Q (p; F)) \leq F \left( y_{n + 1} \right) \leq F
\left( y_{n} \right)\) for all \(n \in \mathbb{N}\).
By right-continuity of \(F\), \(\lim_{n \to \infty} F \left( y_{n} \right) = F
(Q (p; F))\).
Set \(\varepsilon = \frac{1}{2} [p - F (Q (p; F))] > 0\).
Hence, there eyists \(N_{\varepsilon} \in \mathbb{N}\) such that
\begin{equation}
  F \left( y_{n} \right) - F (Q (p; F)) = \left| F \left( y_{n} \right) - F (Q
  (p; F)) \right| < \varepsilon = \frac{p - F (Q (p; F))}{2} \qquad \forall n
  \geq N_{\varepsilon}.
\end{equation}
Therefore, \(F \left( y_{n} \right) < \frac{1}{2} [p + F (Q (p; F))] < p\).
Pick any \(n \geq N_{\varepsilon}\),
Then by construction, \(y_{n} > Q (p; F)\) and since \(F \left( y_{n} \right) <
p\), \(y_{n} < y\) for any \(y\) such that \(F (y) \geq p\).
Thus we have
\begin{equation*}
  Q (p; F) < y_{n} \leq \inf \left\{ y \in \mathbb{R} : F (y) \geq p \right\} =
  Q (p; F),
\end{equation*}
which is a contradiction.
\end{proof}

\begin{lemma}
\label{lem--FQp-equal-p-continuity}
If \(F\) is continuous at \(Q (p; F)\), then \(F (Q (p; F)) = p\).
\end{lemma}

\begin{proof}[Proof of \Cref{lem--FQp-equal-p-continuity}]
It suffices to show continuity of \(F\) implies \(F (Q (p; F)) \leq p\) since
we know \(F (Q (p; F)) \geq p\) for any cdf \(F\) by \Cref{lem--FQ-geq-p}.
We do this by contrapositive, i.e. if \(F (Q (p; F)) > p\),
then \(F\) has a discontinuity at \(Q (p; F)\).
It is sufficient to find a sequence \(\left\{ y_{n} \right\}\) such that
\(\lim_{n \to \infty} y_{n} = Q (p; F)\), but \(\limsup_{n \to \infty} F
\left( y_{n} \right) \neq F (Q (p; F))\).

Since \(Q (p; F) := \inf \left\{ y \in \mathbb{R} : F (y) \geq p
\right\}\), we know that if \(y < Q (p; F)\), then \(F (y) < p\).
Take a sequence \(\left\{ y_{n} \right\}\) such that \(y_{n} \uparrow Q (p;
F)\), i.e. \(y_{n} \leq y_{n + 1} < Q (p; F)\) for every \(n \in \mathbb{N}\)
and \(\lim_{n \to \infty} y_{n} = Q (p; F)\).
A concrete eyample would be for instance \(y_{n} := Q (p; F) - (1 / n)\).
For each \(n \in \mathbb{N}\), it follows that \(F \left( y_{n} \right) < p\)
since \(y_{n} < Q (p; F)\).
This implies \(\limsup_{n \to \infty} F \left( y_{n} \right) \leq p < F (Q (p;
F))\).
It therefore follows that \(F\) is discontinuous at \(Q (p; F)\).
\end{proof}


%%% Local Variables:
%%% mode: LaTeX
%%% TeX-master: "../note-clt-sample-quantile"
%%% End:

% LocalWords:  cdf ecdf
