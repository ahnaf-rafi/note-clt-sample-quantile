%! TEX root = ../note-clt-sample-quantile.tex

\section{Proof of
\texorpdfstring{\Cref{thm--clt-quantile}}{Theorem
\ref{thm--clt-quantile}}}
\label{sec--prf--thm--clt-quantile}

Differentiability of \(F\) at \(\theta_{p}\) with derivative \(f \left(
\theta_{p} \right)\) implies both left- and right-differentiability with left-
and right-derivatives both equal to \(f \left( \theta_{p} \right)\) (see
\Cref{rem--diff-equiv-bi-dir-diff}).
Hence \eqref{eqn--clt-quantile} is
implied by the combination of \eqref{eqn--clt-quantile-0} and
\eqref{eqn--clt-quantile-non0}.
So it suffices to prove \eqref{eqn--clt-quantile-0} and
\eqref{eqn--clt-quantile-non0}.
To start, recall from \eqref{eqn--sample-quantile-law-def} that
\begin{equation*}
  H_{p, n} (t) = \Pr \left\{ \widehat{\theta}_{p, n} \leq \theta_{p} +
  \frac{t}{\sqrt{n}} \right\}.
\end{equation*}
We start with a lemma which allow us to relate this to the ecdf
\(\widehat{F}_{n}\).

\Cref{lem--p-leq-Fy-iff-Qp-leq-y} shows in \eqref{eqn--p-leq-Fy-iff-Qp-leq-y}
that \(\widehat{\theta}_{p} \leq y\) if and only if \(\widehat{F}_{n} (y) \geq
p\).
Combine this with \eqref{eqn--sample-quantile-law-def} by setting \(y =
\theta_{p} + (t / \sqrt{n})\) to see that
\begin{equation*}
  H_{p, n} (t) = \Pr \left\{ \widehat{F}_{n} \left( \theta_{p} +
  \frac{t}{\sqrt{n}} \right) \geq p \right\}.
\end{equation*}
For brevity, rewrite this as
\begin{equation}
  H_{p, n} (t) = \Pr \left\{ \widehat{F}_{n} \left( y_{p, n} (t) \right) \geq p
  \right\},
  \quad \text{where} \quad
  y_{p, n} (t) = \theta_{p} + \frac{t}{\sqrt{n}}.
  \label{eqn--sample-quantile-sampling-dist-1}
\end{equation}
We know that \(n \widehat{F}_{n} \left( y_{p, n} (t) \right) \sim
\mathrm{Binomial} \left( n, F \left( y_{p, n} (t) \right) \right)\).
To help characterize limits of \(H_{p, n}\) via
\eqref{eqn--sample-quantile-sampling-dist-1},
\Cref{thm--bin-clt-normal} provides a central limit theorem for Binomial
distributions whose trial success probabilities can potentially change with the
number of trials.

\begin{theorem}
\label{thm--bin-clt-normal}
Let \(Z \sim \N (0, 1)\), \(\Phi (z) := \Pr \{Z \leq z\}\), \(\left\{ p_{n}
\right\}\) be a sequence in \([0, 1]\)
and \(S_{n} \sim \mathrm{Binomial} \left( n, p_{n} \right)\).
\begin{equation}
  \text{If} \quad
  \lim_{n \to \infty} \frac{1}{\sqrt{n p_{n} \left( 1 - p_{n} \right)}} = 0,
  \quad \text{then} \quad \frac{S_{n} - n p_{n}}{\sqrt{n p_{n} \left( 1 - p_{n}
  \right)}} =: Z_{n} \rightsquigarrow Z.
  \label{eqn--bin-clt-normal}
\end{equation}
Hence, if for some \(p_{\ast} \in (0, 1)\) and \(\Delta \in \mathbb{R}\),
\(p_{n} = p_{\ast} + n^{- 1 / 2} \Delta + o \left( n^{- 1 / 2} \right)\),
then
\begin{equation}
  \lim_{n \to \infty} \Pr \left\{ \frac{S_{n}}{n} \geq p_{\ast} \right\} = \Phi
  \left( \frac{\Delta}{\sqrt{p_{\ast} \left( 1 - p_{\ast} \right)}} \right).
  \label{eqn--bin-clt-normal-Sn}
\end{equation}
\end{theorem}

\begin{proof}
See \Cref{sec--prf--thm--bin-clt-normal}.
\end{proof}

\subsection{Proof of
\texorpdfstring{\eqref{eqn--clt-quantile-0}}{(\ref{eqn--clt-quantile-0})}
in \texorpdfstring{\Cref{thm--clt-quantile}}{Theorem \ref{thm--clt-quantile}}}

In this case, \(t = 0\) and so \(y_{p, n} (t) = \theta_{p}\) is constant in
\(n\).
Set
\begin{equation}
  S_{n} = n \widehat{F}_{n} \left( \theta_{p} \right) \quad \text{and} \quad
  p_{n} = F \left( \theta_{p} \right).
  \label{eqn--Sn-pn-application-def-0}
\end{equation}
The hypotheses for \eqref{eqn--bin-clt-normal} and
\eqref{eqn--bin-clt-normal-Sn} are satisfied with \(p_{n} = p_{\ast} = F \left(
\theta_{p} \right)\) and \(\Delta = 0\).
Furthermore, since \(p = F \left( \theta_{p} \right)\) by hypothesis to
\eqref{eqn--clt-quantile-0}, from \eqref{eqn--sample-quantile-sampling-dist-1},
\eqref{eqn--bin-clt-normal-Sn}, and \eqref{eqn--Sn-pn-application-def-0} we get
\(\lim_{n \to \infty} H_{p, n} (0) = \Phi (0) = \frac{1}{2}\) which proves
\eqref{eqn--clt-quantile-0}.
\qed

\subsection{On necessity of the condition \texorpdfstring{\(p = F \left(
\theta_{p} \right)\)}{p = F (theta p)}}

Continue to consider the case \(t = 0\), but suppose that \(p \neq F \left(
\theta_{p} \right)\).
By \Cref{lem--FQ-geq-p}, \(p \leq F \left( \theta_{p} \right)\).
So if \(p \neq F \left( \theta_{p} \right)\), then it must be the case that
\(p < F \left( \theta_{p} \right)\).
We can still apply \eqref{eqn--bin-clt-normal} in \Cref{thm--bin-clt-normal}
still applies with \(S_{n} = n \widehat{F}_{n} \left( \theta_{p} \right)\)
and \(p_{n} = F \left( \theta_{p} \right)\).
Then combine \eqref{eqn--sample-quantile-sampling-dist-1} and
\eqref{eqn--bin-clt-normal} as follows:
\begin{align*}
  H_{p, n} (0) =
  & \, \Pr \left\{ \widehat{F}_{n} \left( \theta_{p} \right) \geq p
  \right\} = \Pr \left\{ \frac{n \left( \widehat{F}_{n} \left( \theta_{p}
  \right) - F \left( \theta_{p} \right) \right)}{\sqrt{n F \left( \theta_{p}
  \right) \left( 1 - F \left( \theta_{p} \right) \right)}} \geq \frac{n \left( p
  - F \left( \theta_{p} \right) \right)}{\sqrt{n F \left( \theta_{p} \right)
  \left( 1 - F \left( \theta_{p} \right) \right)}} \right\} \\
  =
  & \, \Pr \left\{ Z_{n} \geq \frac{\sqrt{n} \left( p - F \left( \theta_{p}
  \right) \right)}{\sqrt{F \left( \theta_{p} \right) \left( 1 - F \left(
  \theta_{p} \right) \right)}} \right\}.
\end{align*}
But by \(p < F \left( \theta_{p} \right)\),
\begin{equation*}
  \lim_{n \to \infty} \frac{\sqrt{n} \left( p - F \left( \theta_{p}
  \right) \right)}{\sqrt{F \left( \theta_{p} \right) \left( 1 - F \left(
  \theta_{p} \right) \right)}} = - \infty,
\end{equation*}
and so
\begin{align*}
  \lim_{n \to \infty} H_{p, n} (0) =
  & \, \lim_{n \to \infty} \Pr \left\{ Z_{n} \geq \frac{\sqrt{n} \left( p - F
  \left( \theta_{p} \right) \right)}{\sqrt{F \left( \theta_{p} \right) \left( 1
  - F \left( \theta_{p} \right) \right)}} \right\} = 1.
\end{align*}
Of course since \(1 \geq H_{p, n} (t) \geq H_{p, n} (0)\) if \(t > 0\),
\(\lim_{n \to \infty} H_{p, n} (t) = 1\) for \(t > 0\) as well.
That is, if \(p \neq F \left( \theta_{p} \right)\), we cannot get a
non-degenerate Gaussian limit for \(\sqrt{n} \left( \widehat{\theta}_{p, n} -
\theta_{p} \right)\).

Since we focus only on the cases where \(H_{p, n}\) limits to a non-degenerate
Gaussian, it is therefore necessary to impose the restriction \(p = F \left(
\theta_{p} \right)\).
To ensure this holds for all \(p \in (0, 1)\) we can maintain an assumption of
continuity of \(F\) at \(\theta_{p}\) as in \Cref{lem--FQp-equal-p-continuity}.

\subsection{Proof of
\texorpdfstring{\eqref{eqn--clt-quantile-non0}}{(\ref{eqn--clt-quantile-non0})}
in \texorpdfstring{\Cref{thm--clt-quantile}}{Theorem \ref{thm--clt-quantile}}}

Here we set
\begin{equation}
  S_{n} = n \widehat{F}_{n} \left( y_{p, n} (t) \right), \quad
  p_{n} = F \left( y_{p, n} (t) \right), \quad p_{\ast} = p = F \left(
  \theta_{p} \right), \quad \text{and} \quad \Delta = f \left( \theta_{p}
  \right) t.
  \label{eqn--Sn-pn-application-def-non0}
\end{equation}
Then \eqref{eqn--sample-quantile-sampling-dist-1} becomes
\begin{equation*}
  H_{p, n} (t) = \Pr \left\{ \frac{S_{n}}{n} \geq p_{\ast} \right\}.
\end{equation*}
Hence we wish to apply \eqref{eqn--bin-clt-normal-Sn} and so we have to show
\(p_{n}\), \(p_{\ast}\) and \(\Delta\) as defined in
\eqref{eqn--Sn-pn-application-def-non0} satisfy
\begin{equation}
  \lim_{n \to \infty} \sqrt{n} \left( p_{n} - p_{\ast} - \frac{\Delta}{\sqrt{n}}
  \right) = 0.
  \label{eqn--bin-clt-normal-Sn-hyp-1}
\end{equation}
To that end, let \(\delta_{n} = t / \sqrt{n}\) and note that since \(t \neq 0\),
\begin{align*}
  \sqrt{n} \left( p_{n} - p_{\ast} - \frac{\Delta}{\sqrt{n}}
  \right) =
  & \, \sqrt{n} \left( F \left( \theta_{p} + \delta_{n} \right) - F \left(
  \theta_{p} \right) - f \left( \theta_{p} \right) \delta_{n} \right) \\
  =
  & \, t \cdot \frac{F \left( \theta_{p} + \delta_{n} \right) - F \left(
  \theta_{p} \right) - f \left( \theta_{p} \right) \delta_{n}}{\delta_{n}}.
\end{align*}
Therefore, under either hypothesis of directional differentiability (since the
sign of \(\delta_{n}\) and \(t\) are equal),
\begin{equation*}
  \lim_{n \to \infty} \sqrt{n} \left( p_{n} - p_{\ast} - \frac{\Delta}{\sqrt{n}}
  \right) = t \lim_{n \to \infty} \cdot \frac{F \left( \theta_{p} + \delta_{n}
\right) - F \left(
  \theta_{p} \right) - f \left( \theta_{p} \right) \delta_{n}}{\delta_{n}} = 0,
\end{equation*}
which shows \eqref{eqn--bin-clt-normal-Sn-hyp-1}.
Apply \eqref{eqn--bin-clt-normal-Sn} with
\eqref{eqn--Sn-pn-application-def-non0} to get
\begin{equation*}
  \lim_{n \to \infty} H_{p, n} (t) = \lim_{n \to \infty} \Pr \left\{
  \frac{S_{n}}{n} \geq p_{\ast} \right\} = \Phi \left(
  \frac{\Delta}{\sqrt{p_{\ast} \left( 1 - p_{\ast} \right)}} \right) = \Phi
  \left( \frac{f \left( \theta_{p} \right) t}{\sqrt{p (1 - p)}} \right),
\end{equation*}
which is exactly \eqref{eqn--clt-quantile-non0}.
\qed

%%% Local Variables:
%%% mode: LaTeX
%%% TeX-master: "../note-clt-sample-quantile"
%%% End:

% LocalWords:  cdf ecdf
